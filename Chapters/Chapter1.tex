% Chapter 1

\chapter{Introduction}
\label{Chapter1} 

ECG-ira stands for ECG (electrocardiogram) instantaneous responsive analyzer and it is an android application developed for the acquisition, visualization and analysis of the electrocardiogram signals. The application acquires and stores  the record from the zecg acquisition device  (using the zecg device format) property of the Politecnico of Milan but it can also open and visualize other ecg format from other  devices. Ecg-ira is part of a greater and long term project (zecg itself was part of a first step). ECG-ira aims to exploit the reliability and performance of a complex software for electrocardiogram signal acquisition,  visualization  and processing within a smartphone device through a mobile application. Apart of the analysis algorithm already implemented during a past thesis (by Diego Ulisse Pizzagalli), the application was designed and implemented from scratch. During the process we dealt with the devices limitation in term of performance power and limited memory. We overcome many issues, related also to small and different screen size and density of pixel, due to the great number of different devices currently on the market. We had to exploit the multithreading and efficient memory usage strategies in order to achieve responsiveness and fulfill the medical requirement for an ECG compliant application. The result is an application which is easy to use because it follows all the best design principles according to the official Google guidelines for responsive UI and UX. By taking advantage of the multithreading capabilities and the usage of all the available cores into a device, we achieved an application which performs fast and well.\\
This thesis is structured as follows: 
\begin{itemize}
	\item After this introduction we give a brief but detailed overview about the heart, its functionalities and how an electrocardiogram is related to it by explaining starting from what it is an ecg to how heart electrical signals are detected and read.
	\item In the State of Art chapter we show the panorama of the actual devices for an ecg acquisition and some available applications on the market which allow to open and read an ecg record.
	\item Then it follows the Objective chapter in which we describe and explain the goal of this thesis work.
	\item The Requirements chapter is where we list functional and nonfunctional sets of features which came up during the planning phase. The final result should be compliant with all of them.
	\item The Problem chapter deals with all the issues related to the project development. Here we discussed about the problem of choosing a development platform with respect to another, the hardware limitation on a smartphone device and the problems strictly related to the ecg signals.
	\item The Solution choices chapter is where we explain and provide our reasonings to the implementation choices, from the platform choice to the programming language choice to why we decided to avoid using drawings libraries preferring a custom and proprietary implementation.
	\item In the System architecture chapter we provide a general overview of the project structure for both the acquisition device implementation and architecture both the mobile application structure and main functionalities.
	\item In the Implementation details chapter we described all the main components and Java classes. Here we provide also hints for future implementations and adaptations.
	\item In the Final result chapter we propose some screenshots, with descriptive captions and descriptions of the main screens of the application; then we go through a performance analysis of application by observing how it affects the memory, the cpu and the response time of some portion of the code. The tests were conducted on many different devices but only data from the older devices (2011) and the newest (November 2015) are compared.
	\item In the Conclusion chapter we sum up the overall results giving an overview of what has been done and providing some post considerations about the work done.
	\item In the Future works chapter we provide hints for further implementations in order to make ECG-ira really an all in one solution as an ECG mobile application. We also put our consideration on the future of the mobile software with respect to the desktop one, according to the new trends and the fact that the mobile environments is penetrating even more in our daily lives and can really positively change the way patients are connected with the health care providers.
\end{itemize}