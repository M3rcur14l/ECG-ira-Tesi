% Chapter 1

\chapter{Introduction}
\label{Chapter1} 

ECG-ira stands for ECG (electrocardiographic) instantaneous responsive analyzer and it is an Android application developed for the acquisition, visualization and analysis of the electrocardiographic signals. The application acquires and stores the leads from the ZEcg acquisition device  (using the ZEcg device format) property of the Politecnico of Milan: the app can also open and visualize other ECG formats from other devices.

Ecg-ira is part of a greater and long term project (ZEcg itself was part of a first step). ECG-ira aims at exploiting the reliability and performance of a complex software for electrocardiographic signal acquisition, visualization  and processing running over a smartphone device through a mobile application. Apart of the analysis algorithm which was already implemented during a previous thesis (by Diego Ulisse Pizzagalli and Simone Battaglia~\cite{ref3}), the application of the current thesis was designed and implemented from scratch. During the process we dealt with the devices limitations, in term of performance power and limited amount of memory. For instance, we had to deal with the exclusive memory allocated for an app by the OS. This memory allocation is called heap and can vary a lot, depending on the devices. In some low-end devices, it can be limited to 16MB.

We overcame many issues, related also to small and different screen sizes and density of pixel, due to the great number of different devices currently on the market. We had to exploit the multithreading and efficient memory usage strategies in order to achieve responsiveness and fulfill the medical requirement for an ECG compliant application.

The result is an application which is easy to use, because it follows all the best design principles as specified by the official Google guidelines for responsive UI and UX~\cite{ref27}. By taking advantage of the multithreading capabilities and the usage of all the available cores into a device, we achieved an application which performs fast and well.

This thesis is structured as follows: 
\begin{itemize}
	\item After the introduction we give a brief but detailed overview about the heart, its functionalities and how an electrocardiogram is related to it. We describe the ECG and the electrical signals it detects.
	\item In the State of Art chapter we show the panorama of the actual devices for ECG acquisition and some available applications on the market which allow one to open and read an ECG record.
	\item The Objective chapter we describes and explains the goal of this thesis work.
	\item The Requirements chapter lists all the functional and the nonfunctional sets of features which came up during the planning phase. The final result must comply all of them.
	\item The Problem chapter deals with all the issues related to the project development. We discuss the issues about the development platform to be adopted, the hardware limitations on a smartphone device and the problems strictly related to the ECG signals.
	\item The Solution Choices chapter describes our implementation choices, from the platform choice to the programming language choice to the proprietary implementation of a drawing library.
	\item The System Architecture chapter provides the reader with a general overview of the project structure, considering the acquisition device, the architecture of the mobile application, and its main functionalities.
	\item The Implementation Details chapter describes all the major components and the Java classes. This chapter also provides the reader with hints for future implementations and enhancements.
	\item The Results chapter proposes some screenshots from the application; the chapter also describes some preliminary performance analysis considering the required amount of memory, the CPU workload, and the response time of some portions of the code. The tests were conducted on many different devices, but data from the older devices (2011) and the newest (November 2015) one are compared, only.
	\item The Conclusion chapter sums up the overall results and an overview of the work done.
	The chapter also provides the reader with hints for further implementations, to make ECG-ira really an all in one solution as an ECG mobile application. We also discuss the future of mobile software with respect to the desktop one, according to the trends and increasing wide-spread diffusion of mobile environments, especially in the health care scenario.
\end{itemize}