%Chapter4
\chapter{Objective}
\label{Chapter4} 

\section{Preface}
For a clear understanding of the next chapters we will make use of some terms listed below with the proper meanings:
\begin{enumerate}
	\item Mobile application: it is a software running on smartphones and tablets
	\item Desktop application: it is a software running on desktop pcs or notebooks
	\item Acquisition device: named ZEcg, it is the device (hardware) used to acquire the ECG signal from the electrodes connected to a patient body.
\end{enumerate}

\section{Fully functional medical mobile app as replacement to desktop app}
The main purpose for this thesis is to develop a medical  mobile application as replacement to an original desktop application. The application needs to be standalone and independent from other software, still it can share its content and integrate other software content.\\
As a starting point we planned to reproduce all the desktop features such as the connection between the application and the remote device ZEcg for the ECG signal acquisition. It should also save the ECG records inside the mobile device, plot the signals and run arrhythmia recognition algorithms on them.  We are aware that the user experience is different from a desktop one due to the differences in capabilities and functionalities. Having in mind these differences, we did not try to reproduce the desktop experience. We developed instead the application having a mobile experience at first position, following the standards of mobile application designs and principles. We took advantage of the new and latest technologies mounted on the new smartphones, trying to provide to the end user the best in term of user experience, performance and application design. The main difficulty is probably to redesign and re-imagine the desktop feature from a mobile point of view. For example, if a desktop application usually makes use of keyboard and mouse, inside a modern mobile application there is only the touch input as user interaction. The differences in term of screen size, memory and CPU performance matters and should always be kept in mind during the initial planning phase. We will deal with these and others limitations, trying to achieve the best results and performances. \\
We believe this application can be really a replacement to a desktop application as the technology trend points to future devices with better performances in term of lower power consumption and higher operational capabilities.\cite{ref2}


