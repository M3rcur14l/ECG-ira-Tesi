% Chapter 11

\chapter{Conclusions}\label{Chapter11}

ECG-ira is a reliable application for mobile devices for the acquisition, visualization and analysis of electrocardiographic recordings. The application is designed to be flexible, customizable, and easy to extend to additional features. With the increasing trend of the healthcare business, ECG-ira is a good help both for non-professional and for healthcare professionals, enabling them to monitor and read ECG recordings at any time and everywhere. In its current state, ECG-ira is compliant with the ZEcg acquisition device, only: however, ECG-ira can be easily enhanced to cope with more devices by exploiting its highly customizable settings. ECG-ira can be connected to other open electrocardiographic recordings and file formats.

The application was designed mobile first, considering all the trade-off and limitations of mobile devices. We exploited the best performance and provided the best user experience according to the last UI design and development patterns, independently from which (compatible) device the application is executed on.

In the near future, mobile applications will become more and more widespread, and will also replace desktop applications, as people moved their habit from desktop PCs to smartphone devices. The increasing phenomenon of the IoT (Internet of Things) will bring smart objects connected all to each other and the number of smart devices and mobile application for the healthcare are becoming of common use for every consumer. The use of devices to measure an athlete's performance has been quite common for some years in the professional ambit environment: nowadays, these technologies are more and more affordable also to a wider arena of non professional athletes and sport practitioners in general.

As people became more and more familiar with mobile applications, ECG-ira is a complete solution for electrocardiographic recording acquisition, visualization and analysis. Due to its ease of use, taking an electrocardiographic recording does no longer require a professional healthcare expert (nurse of physician), and most people are already familiar with the smartphone devices.


\section{Future works}

ECG-ira represents a starting point to build a complete software tool for the acquisition, visualization and analysis of electrocardiographic recordings. More ECG formats can be added, beyond those already implemented: ZEcg format (format ``0'' and format ``1'') and the MIT/BIH format (format 212). Among the possible formats to enhance the performances of the application, we mention here the format ``16'' as defined by the American Heart Association (and also used by the Poli/C\'a Granda ECG/VCG database) and the SCP standard~\cite{ref28}.

More analysis techniques can also be added, such as ischemia detection techniques. Those techniques are mainly devoted to detect insufficient flow of blood to the heart, which may lead to acute myocardial infarction (IMA).

More enhancements include the capability of remote transmission of the recordings to an external web server: this could be the starting point for a revolutionary way of home-monitoring for patients suffering from chronic heart diseases.
