% Chapter 11

\chapter{Conclusions}
\label{Chapter11}

ECG-ira is a reliable software for mobile devices for the acquisition, visualization and analysis of Electrocardiogram records. The application is designed to be flexible and easy to extend with additional features. With the increasing trend of the self healthcare business, ECG-ira is the ideal solution for both amatorial both healthcare professionals to monitor and read and ECG at any time wherever they go. Right now ECG-ira is compliant only with the ZEcg acquisition device, but it is not limited to it. Thanks to the highly parameterized settings, the application can “easily” connect to other acquisition devices if specifications are known. ECG-ira can be used to open other electrocardiogram records and files format other than the ZEcg.\\
The application was designed mobile first, considering all the thread-off and limitation of mobile devices. We exploited the best performance and provided the best user experience according to the last UI design and development patterns independently from which (compatible) device the application is executed on.

\section{Future works}
ECG-ira represents a starting point to build an amazing software for the acquisition, visualization and analysis of all electrocardiogram data. The state of art of the application is limited to the acquisition of ECG record from the ZEcg device and the visualization and analysis of two formats: the ZEcg format (format 1) and the MIT/BIH format (format 212). The next  steps to make ECG-ira an all in one solution for the electrocardiogram visualization and analysis is to develop other ECG format decoder. We implemented the software in a way that it’s required only to write down a format decoder class implementing a few interfaces.

\section{Mobile as replacement to desktop}
We see the future in the mobile software which will replace the old desktop’s one for certain usage. As people moved most of their habit related to software and program usage from desktop pc to smartphone devices, we also believe that in a close future we will start using our own devices as healthcare/medical tools. The increasing phenomena of the IoT will bring smart objects connected all to each other and the number of smart devices and mobile application for the self healthcare are becoming of common use for the consumers. Right now the most used devices are the smartbands for activity recording such as fitness activities (running, gym, athletics). The use of devices to measure an athlete's performance was rather common in the agonistic ambit, nowadays we can see that these technologies became more affordable also to a wider plateau of non agonistic athletes and sport practitioners in general. The great advantage to have a medical application is of course its great portability and mobility. The user can finally be free from constant local checks in by moving to a hospitals or medical infrastructures to just acquire an ECG. With just a basic support from a nurse or a pharmacist or by himself after a proper training, he can measure and acquire the records by himself and after that send it to his doctor who may check it in real time.

\section{Mobile as first medical solution}
As people became more and more familiar with mobile application, with ECG-ira we want to propose a complete solution for electrocardiogram acquisition, visualization and analysis. Taking an electrocardiogram record doesn’t  strictly require a professional Healthcare expert and most of the people are already familiar with the smartphone devices.\\
Adults chiefly result having more familiarity with mobile devices and tablets than with the old desktop pc. A mobile software can bring great innovation and benefit to the Healthcare experts by connecting the professionals with patients in the same way they are already connected with the rest of the world. Patients can skip long queues just for a fast check in and can receive an ECG report about his status, directly on his smartphone. The doctor instead can save time and take care of more patients because he can physically assist less persons and virtually the other ones by reading the records received. Another advantage for other is that anybody will be anymore limited to a place in order to assist or being assisted. A mobile software solution applied to medical ambit can then definitely change  and affect the interaction between the patients and the health care providers simplifying the process and reducing the time in the between. We believe that in a next future, it will play a fundamental if not first role and first choices for hospitals,  clinics and in the private sector.