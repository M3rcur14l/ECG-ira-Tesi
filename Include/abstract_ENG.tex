\chapter{Abstract}

In recent years, the healthcare market has maintained a strong trend of interest and experienced a rapid development. This development has taken advantage of the potential made available by the latest Mobile technologies: there are many companies that have specialized in the creation of proprietary devices which carry out a particular function, ranging from those that would provide support for the world of fitness, such as analysis of body mass, or heartbeat, to those focused in the medical field, such as blood pressure sensors or glucose level. Many of these have potential as the high characteristic of interfacing with the user, through the use of mobile apps that are synchronized with these devices and that analyze the data in order to present them to the user in the most accessible way possible.\\
Our work aims to fit in this area, more specifically in the medical one. Our project was born from the work of some students who have gone before, and presents itself as an evolution in a modern way. This has allowed us to have, as a starting point, a device and a PC software (previously developed) to interface with. In particular, the first is an ECG (electrocardiogram) signal acquisition device, called ZEcg. Its special feature is the small size and its connection interface, which uses the Bluetooth technology, being so wireless.  The software however, comes earlier, originally as a PC software for visualization and analysis of ECG arrhythmia. Later it was adapted, in order to add as functionality the real-time acquisition from the ZEcg device of ECG records.\\
Leveraging the ZEcg device and the previous analysis software algorithms, the project has as its goal the creation of a mobile app for smartphones and tablets, to become a real support for a physician, and therefore provide all necessary functionality to capture, view and analyze an ECG. In addition, the app will need to be designed for maximum extensibility and compatibility with more mobile devices. The responsive and speed features the application will have as its first objective, took us to the name "ECG-ira", which stands for "ECG-instantaneous responsive analyzer".\\
In the course of the chapters we will analyze the problems encountered and the implementation choices that have led to the creation of the final product.