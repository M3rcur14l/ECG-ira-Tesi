\chapter{Abstract}

In recent years, the healthcare market showed a continuously increasing trend of interest, and experienced a rapid development. This development has taken advantage of the potential made available by the latest mobile technologies: many companies created proprietary devices suitably designed for specific purposes, ranging from support to the world of fitness, such as analysis of body mass, or heartbeat, to those focused on the medical field, such as blood pressure sensors or glucose level meters. Many of these devices may easily interface with mobile apps, synchronizing and analyzing collected data, presenting them to the user in the most accessible way.

Our work fits in this area, more specifically in the medical one. Our work belongs to a wider project, which saw the work of some students who came first: it presents itself as an evolution in a more modern way. During our work, we used, as a starting point, a device and a PC software (previously developed) to interface with. The device is an ECG (electrocardiographic) signal acquisition module, called ZEcg: its main features are its small size and its connection interface, which uses the Bluetooth technology, being so wireless.  The software, originally deployed on PCs, aims at acquiring and visualizing in real time the ECG signals received from ZEcg, and at identifying possible arrhythmic events therein.

Leveraging the ZEcg device and the previous analysis algorithms, our work lead to the development of a mobile app running on smartphones and on tablets: the result is a real support for the physician on the field, and therefore provides all the necessary functionalities to capture, view and analyze an ECG recording.

The app was designed for maximum flexibility, extensibility and compatibility with more mobile devices. The responsive and speed features the application has as its first objective, took us to the name ``ECG-ira'', which stands for ``ECG-instantaneous responsive analyzer''.

The following chapters will analyze the problems encountered and the implementation choices that have led to the creation of the final system.