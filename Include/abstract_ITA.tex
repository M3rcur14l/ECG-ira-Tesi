\chapter{Sommario}

Negli ultimi anni, il mercato della healthcare ha mostrato una continua tendenza di interesse, e subito un rapido sviluppo. Questo sviluppo ha sfruttato tutte le potenzialità rese disponibili dalle ultime tecnologie mobile: molte aziende hanno creato dispositivi proprietari opportunamente progettati per scopi specifici, andando dal supporto al mondo del fitness, come l’analisi della massa corporea o del battito cardiaco, a quelli focalizzati nel campo medicale, come i sensori di pressione sanguigna o i glucometri. Molti di questi dispositivi possono facilmente interfacciarsi con mobile app, sincronizzando e analizzando i dati raccolti, presentandoli all'utente nel modo più accessibile.

Il nostro lavoro fa parte di un progetto più ampio, che ha visto l’operato di alcuni studenti che sono venuti prima: esso si presenta come una sua moderna evoluzione. Durante il nostro lavoro, abbiamo utilizzato, come punto di partenza, un dispositivo e un software per PC (sviluppato in precedenza) con i quali ci siamo interfacciati. Il dispositivo è un modulo di acquisizione di segnale ECG (elettrocardiogramma), chiamato ZEcg: le sue principali funzionalità sono le ridotte dimensioni e la sua interfaccia di connessione, che sfrutta la tecnologia bluetooth, e quindi senza fili. Il software, originariamente sviluppato per PC, ha lo scopo di acquisire e visualizzare in tempo reale il segnale ECG ricevuto da ZEcg, e identificare possibili eventi aritmici all’interno di esso.

Sfruttando il dispositivo ZEcg ed i precedenti algoritmi di analisi, il nostro lavoro ci ha portato allo sviluppo di un'applicazione mobile per smartphone e tablet: il risultato è un vero e proprio supporto per il medico in ambito lavorativo, fornendo perciò tutte le funzionalità necessarie per acquisire, visualizzare e analizzare una registrazione ECG. 

L'applicazione è stata progettata per la massima flessibilità, estensibilità e compatibilità con più dispositivi mobili. Le caratteristiche di adattamento e rapidità che l’applicazione avrà come primi obbiettivi, ci hanno portato al nome ``ECG-ira'', il quale sta per ``ECG-instantaneous responsive analyzer''. 

Nei prossimi capitoli si analizzeranno i problemi riscontrati e le scelte implementative che hanno portato alla realizzazione del sistema finale.