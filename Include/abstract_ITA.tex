\chapter{Sommario}

Negli ultimi anni il mercato della healthcare ha mantenuto una forte tendenza di interesse e subito un rapido sviluppo. Questo sviluppo ha sfruttato tutte le potenzialità rese disponibili dalle ultime tecnologie mobile: sono numerose le aziende che si sono specializzate nella creazione di dispositivi proprietari che svolgessero una determinata funzione: si va da quelli che fornissero supporti per il mondo del fitness, come l’analisi della massa corporea o del battito cardiaco, a quelli focalizzati nel campo medicale, come i sensori di pressione sanguigna o di livello di glucosio. Molti di questi hanno come potenzialità l’alta caratteristica di interfacciamento con l’utente, attraverso l’uso di mobile app che si sincronizzino con questi dispositivi e che analizzino i dati al fine di presentarli all’utente nel modo più accessibile possibile.\\
Il nostro lavoro vuole inserirsi in questo ambito, più specificatamente in quello medicale. Il nostro progetto nasce dal lavoro di alcuni studenti che ci hanno preceduto, e si presenta come una sua evoluzione in chiave moderna. Questo ci ha permesso di avere, come punto di partenza, un dispositivo e un software PC (precedentemente sviluppati) con cui interfacciarci. In particolare, il primo è un dispositivo di acquisizione di segnale ECG (elettrocardiogramma), chiamato ZEcg. La sua particolarità è rappresentata dalle ridotte dimensioni e dal suo interfacciamento di connessione, che sfrutta la tecnologia bluetooth, e quindi senza fili. Il software invece, nasce precedentemente, alle origini come software PC di visualizzazione ed analisi di aritmie di tracciati ECG. Successivamente è stato riadattato, al fine di aggiungere come funzionalità l’acquisizione in tempo reale dal dispositivo ZEcg di tracciati ECG.\\
Sfruttando il dispositivo ZEcg e gli algoritmi di analisi del precedente software, il progetto ha come fine la creazione di una mobile app per smartphone e tablet, che diventi un vero e proprio supporto per un medico specializzato, e che quindi fornisca tutte le funzionalità necessarie per acquisire, visualizzare ed analizzare un tracciato ECG. Inoltre l’app dovrà essere progettata per una massima estensibilità e compatibilità con più dispositivi mobile. Le caratteristiche di adattamento e rapidità che l’applicazione dovrà avere come primi obbiettivi, ci hanno portato al nome “ECG-ira”, che sta per “ECG-instantaneous responsive analyzer”.\\
Nel corso dei capitoli si analizzeranno i problemi riscontrati e le scelte implementative che hanno portato alla realizzazione del prodotto finale.